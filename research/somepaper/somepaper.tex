\documentclass[oneside,a4paper]{article}

\usepackage[
    backend=biber,
    style=ieee,
    sortlocale=en_AU,
    natbib=true,
    url=false,
    doi=true,
    eprint=false
]{biblatex}
%\addbibresource{somepaper.bib}


\usepackage[utf8]{inputenc}
\usepackage[margin=2cm]{geometry}
\usepackage{amsmath}
\usepackage{amsthm}
\usepackage{amssymb}
\usepackage{graphicx}
\usepackage{algorithm}
\usepackage{algorithmicx}
\usepackage[]{algpseudocode}
\usepackage{hyperref}
\usepackage{color}
\usepackage{hyperref}
\usepackage{pdfpages}
\usepackage{listings}

\theoremstyle{remark}
\newtheorem{claim}{Claim}

\begin{document}
   \title{SugarDAO Technical Paper}
    \author{Jack McPherson}
    \date{2021-06-29}
    \maketitle

    \section{Scoring Rule}
        Let \(\alpha\) and \(\beta\) be the lower and upper bounds on healthy BGL, respectively. Let \(\gamma\) be the ideal BGL itself.
        
        Additionally, we use \(d\) to denote the Euclidean distance function. Due to the univariate nature of our setting, we have the following familiar definition,

        \[
            \forall x, y\in\mathbb{R}, d\left(x, y\right)=|x-y|
        \]

        Consider the function, \(p\), defined as follows:

        \[
            p\left(b\right)=\left.
                \begin{cases}
                    \frac{d\left(b, \gamma\right)}{d\left(\alpha, \gamma\right)}, & \text{if } b<\gamma\\
                    \frac{d\left(b, \gamma\right)}{d\left(\beta, \gamma\right)}, & \text{otherwise}\\
                \end{cases}\right.
        \]
        \[
            \Leftrightarrow p\left(b\right)=\left.
                \begin{cases}
                    \frac{|b-\gamma|}{|\alpha-\gamma|}, & \text{if } b<\gamma\\
                    \frac{|b-\gamma|}{|\beta-\gamma|}, & \text{otherwise}\\
                \end{cases}\right.
        \]

        Then the scoring rule is a real-valued function, \(s\), sending blood glucose levels to arbitrary scores.

        \[
            \begin{cases}
                1-p\left(b\right), & \text{if } p\left(b\right)<1\\
                0, & \text{otherwise}\\
            \end{cases}
        \]

        As a concrete example of a particular scoring scheme, consider the following paramaterisation:

        \[
            \begin{cases}
                \alpha=0\text{ mmol}\\
                \beta=17\text{ mmol}\\
                \gamma=7\text{ mmol}
            \end{cases}
        \]


        Then, for a BGL of \(b=6\text{ mmol}\), we have:

        \[
            p\left(6\right)=\frac{|6-7|}{|0-7|}=\frac{|-1|}{|-7|}=\frac{1}{7}\implies s\left(6\right)=1-\frac{1}{7}=\frac{6}{7}
        \]

    %\printbibliography
\end{document}

