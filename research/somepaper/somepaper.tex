\documentclass[oneside,a4paper]{article}

\usepackage[
    backend=biber,
    style=ieee,
    sortlocale=en_AU,
    natbib=true,
    url=false,
    doi=true,
    eprint=false
]{biblatex}
%\addbibresource{somepaper.bib}


\usepackage[utf8]{inputenc}
\usepackage[margin=2cm]{geometry}
\usepackage{amsmath}
\usepackage{amsthm}
\usepackage{amssymb}
\usepackage{graphicx}
\usepackage{algorithm}
\usepackage{algorithmicx}
\usepackage[]{algpseudocode}
\usepackage{hyperref}
\usepackage{color}
\usepackage{hyperref}
\usepackage{pdfpages}
\usepackage{listings}

\theoremstyle{remark}
\newtheorem{claim}{Claim}

\begin{document}
   \title{SugarDAO Technical Paper}
    \author{Jack McPherson}
    \date{2021-06-29}
    \maketitle

    \section{\$SUGAR}
        \subsection{Scoring Rule}
            \subsubsection{Definition}
                Let \(\alpha\) and \(\beta\) be the lower and upper bounds on healthy BGL, respectively. Let \(\gamma\) be the ideal BGL itself. We require that \(\gamma\in [\alpha,\beta]\).
                
                Additionally, we use \(d\) to denote the Euclidean distance function. Due to the univariate nature of our setting, we have the following familiar definition,
        
                \[
                    \forall x, y\in\mathbb{R}, d\left(x, y\right)=|x-y|
                \]
        
                We define a \textit{penalty}, \(p:\mathbb{R}\rightarrow [0, 1]\), defined as follows:
        
                \[
                    p\left(b\right)=\left.
                        \begin{cases}
                            \frac{d\left(b, \gamma\right)}{d\left(\alpha, \gamma\right)}, & \text{if } b<\gamma\\
                            \frac{d\left(b, \gamma\right)}{d\left(\beta, \gamma\right)}, & \text{otherwise}\\
                        \end{cases}\right.
                \]
                \[
                    \Leftrightarrow p\left(b\right)=\left.
                        \begin{cases}
                            \frac{|b-\gamma|}{|\alpha-\gamma|}, & \text{if } b<\gamma\\
                            \frac{|b-\gamma|}{|\beta-\gamma|}, & \text{otherwise}\\
                        \end{cases}\right.
                \]
    
                Then the \textit{score} is a real-valued function, \(s:\mathbb{R}\rightarrow[0,1]\), sending blood glucose levels to arbitrary scores.
        
                \[
                    s\left(b\right)=\begin{cases}
                        1-p\left(b\right), & \text{if } p\left(b\right)<1\\
                        0, & \text{otherwise}\\
                    \end{cases}
                \]
                
                Conceptually, the further away from the ideal BGL \(b\) is, the higher the penalty -- and thus score. If the BGL falls outside of the healthy range (i.e., \(b\notin[\alpha, \beta]\)), then the penalty is maximised at 1. This reduces the score to its \textit{minimum} of 0.
        
        
            \subsubsection{Worked Example}
                As a concrete example of a particular scoring scheme, consider the following paramaterisation:
        
                \[
                    \begin{cases}
                        \alpha=0\text{ mmol}\\
                        \beta=17\text{ mmol}\\
                        \gamma=7\text{ mmol}
                    \end{cases}
                \]
        
                Then, for a BGL of \(b=6\text{ mmol}\), we have:
        
                \[
                    p\left(6\right)=\frac{|6-7|}{|0-7|}=\frac{|-1|}{|-7|}=\frac{1}{7}\implies s\left(6\right)=1-\frac{1}{7}=\frac{6}{7}
                \]

    %\printbibliography
\end{document}

